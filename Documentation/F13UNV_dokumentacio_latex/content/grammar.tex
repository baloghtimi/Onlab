%----------------------------------------------------------------------------
\chapter{Nyelvtan}
%----------------------------------------------------------------------------
A szabályok definiálásához először egy Xtext nyelvtant készítettem. Az alábbi ábrán ennek a szöveges szerkesztője látható, amelyben az előbb példaként felhozott szabályt írtam le. Az elején adhatjuk meg a felhasználókat, vagy egy halmazukat, akikre majd a szabályok vonatkoznak. A szabályokat vagyis rule-okat policyba csoportosíthatjuk, erre megszabhatunk egy alapértelmezett jogosultságot (itt az írás/olvasás tiltott), valamint korlátozó vagy engedélyező tulajdonságot, ami megszabja, hogy az azonos prioritású szabályok között a tiltó vagy az engedélyező a dominánsabb. (Itt pl. mivel korlátozó tulajdonságot szabtunk meg, ha lenne még egy szabály, ami a meglévő engedélyezővel szemben a Zsolti nevű felhasználónak megtiltja a hozzáférést ezekhez az objektumokhoz, akkor az a szabály érvényesülne.) A rule elején megadjuk, hogy melyik felhasználóra vonatkozik, és milyen jogokat ad/tilt. (Itt engedélyezzük Zsoltinak az olvasást/írást.) A végén adhatjuk meg hozzá a már említett prioritást. A rule-on belül egy gráflekérdezésre hivatkozunk. (Az én példám szerint ez adja vissza a szobaseniorokat a színükkel együtt.) Ebből választhatunk ki objektum, attribútum vagy referencia asseteket (itt a szobasenior objektumokat), amelyek halmazát különböző megkötésekkel tovább specifikálhatjuk. (A szobaseniorok közül a kékek kellenek).

\begin{lstlisting}
	user Zsolti                                        // Felhasználó(k)                 
	
	policy camp deny RW by default {                   // Alapértelmezett jogosultság
	
	    rule blueRoomSeniors allow RW to Zsolti {      // Jogosultság + felhasználó
	        from query "roomSeniors"                   // Gráflekérdezés
	        select obj(roomSenior)                     // Asset(ek) kiválasztása
	        where colorType is bound to "BlueColor"    // Megkötés(ek)
	    } with 1 priority                              // Prioritás
	
	} with restrictive resolution                      // Korlátozó/engedélyező tulajdonság
\end{lstlisting}

A hozzáférési szabályok definiálását két extra funkció is kényelmesebbé teszi a felhasználó számára. Az egyik az automatikus formázás, a Ctrl+Shift+F billentyűkombinációt lenyomva az alább látható formára alakul a helyesen beírt szöveg. Ezen kívül szintén Java nyelven ScopeProvider-t is írtam, ami pedig Ctrl+space kombinációt a megfelelő helyeken lenyomva egy listából választhatunk az alternatívák közül. Konkrétan a fel-használók, a létező lekérdezések, és a lekérdezés által visszaadott assetek közül válogathatunk.
